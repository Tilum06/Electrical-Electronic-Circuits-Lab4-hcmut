\section{Volatage Regulator using LM2596}

LM2596 is a voltage regulator mainly used to step down the voltage or to drive load under 3A. It is also known as DC-to-DC power converter or buck converter which is used to step down the voltage from its input supply to the output load. The current goes up during this voltage step down process.

 LM2596 comes with a remarkable load and line regulation. It is available in both versions: fixed output voltage version with 3.3V, 5V, 12V, and customized output version where you can choose the output as per your requirement. This regulator is incorporated with a fixed-frequency oscillator and an internal frequency compensation method.

 The typical connection for LM2596 is proposed by Texas Instrument (TI), as following:

\begin{figure}[H]
    \centering
    \includegraphics[width=0.7\textwidth]{graphics/chap3_question.png}
    \caption{Typical connection for LM2596}
    \label{fig:chap3_question}
\end{figure}    

This circuit is simulated in PSpice in previous lab, and is implemented in Altium Design in this lab.The introduction of this circuit is presented in the video bellow:

\begin{center}
    \underline{\url{https://www.youtube.com/watch?v=57Ra92p3C0k}}
\end{center}


\subsection{Schematic design}

Students are proposed to capture the schematic design in Altium Designer and place the image in this part.

Some hot keys are normally used in the schematic is the space bar, X( horizontal mirror), Y (vertial mirror) and Ctrl + W (place a wire).

The manual is posted in this link:

\begin{center}
    \underline{\url{https://www.youtube.com/watch?v=DGiHsGWPyYw}}
\end{center}

\textbf{Your image goes here}

\begin{figure}[H]
    \centering
    \includegraphics[width=0.7\textwidth]{graphics/chap3_SCH.png}
    \caption{Schematic design}
    \label{fig:chap3_sch}
\end{figure}

\subsection{PCB layout}

Similarly to the schematic, some snap shorts of for the TOP, BOTTOM layers are required in this report. Moreover, several 3D images of your schematic are also required.

\textbf{Your images go here}

\begin{figure}[H]
    \centering
    \includegraphics[width=0.7\textwidth]{graphics/chap3_PCB_2D_bot.png}
    \caption{PCB 2D layout - BOTTOM layer}
    \label{fig:chap3_pcb_2d_bot}
\end{figure}

\begin{figure}[H]
    \centering
    \includegraphics[width=0.7\textwidth]{graphics/chap3_PCB_2D_top.png}
    \caption{PCB 2D layout - TOP layer}
    \label{fig:chap3_pcb_2d_top}
\end{figure}

\begin{figure}[H]
    \centering
    \includegraphics[width=0.7\textwidth]{graphics/chap3_PCB_3D_bot.png}
    \caption{PCB 3D layout - BOTTOM layer}
    \label{fig:chap3_pcb_3d_bot}
\end{figure}

\begin{figure}[H]
    \centering
    \includegraphics[width=0.7\textwidth]{graphics/chap3_PCB_3D_top.png}
    \caption{PCB 3D layout - TOP layer}
    \label{fig:chap3_pcb_3d_top}
\end{figure}
